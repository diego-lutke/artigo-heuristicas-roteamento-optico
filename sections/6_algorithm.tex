\section{Desenvolvimento} \label{sec:algorithm}

Considere uma rede óptica modelada como um grafo $G=(V,E)$,
onde $V$ representa os nós (caixas de emenda ou atendimento) e $E$ representa enlaces (cabos/fibras).
Cada aresta $e \in E$ possui um peso $w(e)$ correspondente ao custo físico (distância, atenuação ou custo de implantação).
O objetivo é determinar o caminho de menor custo entre um nó origem $s$ e um nó destino $t$.

O algoritmo de Dijkstra é empregado para resolver o problema.
O pseudo-código é dado a seguir:

\begin{enumerate}
    \item Inicializar todos os nós com distância infinita, exceto o nó origem $s$ que recebe 0.
    \item Marcar todos os nós como não visitados.
    \item Selecionar o nó não visitado com menor distância atual.
    \item Atualizar as distâncias dos vizinhos, se o novo caminho for mais curto.
    \item Repetir até que todos os nós tenham sido visitados ou que o destino $t$ tenha sido alcançado.
\end{enumerate}

- quando fazer manobras;

...
