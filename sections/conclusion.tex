\section{Conclusão} \label{sec:conclusion}

O presente artigo apresentou uma proposta de heurística para o roteamento
físico em redes ópticas, baseada no algoritmo de Dijkstra e adaptada ao
contexto de alocação de fibras disponíveis. A fundamentação teórica discutiu a
aplicabilidade de algoritmos clássicos de otimização, destacando as limitações
práticas de soluções exatas frente à complexidade NP-difícil do problema. A
heurística desenvolvida propõe um fluxo lógico que combina o cálculo do menor
caminho com a verificação sequencial de disponibilidade de fibras ao longo dos
enlaces, permitindo uma abordagem de roteamento simplificada, porém eficiente.
A formulação apresentada demonstra o potencial de uso de heurísticas para
auxiliar na automação do processo de alocação de recursos ópticos e
planejamento de rotas físicas.

Embora o estudo não tenha contemplado testes empíricos ou simulações, a
proposta estabelece uma base sólida para implementações futuras, podendo ser
expandida para considerar múltiplos comprimentos de onda, métricas de
confiabilidade e critérios de engenharia de tráfego.

Como trabalhos futuros, recomenda-se a validação da heurística em ambiente
simulado, a comparação de desempenho com outros algoritmos de roteamento, e a
integração com ferramentas de gestão de rede óptica.Além disso, ainda como
estudos futuros, recomenda-se a incorporação de restrições adicionais (exemplo:
atenuação, capacidade dos nós, distância máxima), a aplicação de técnicas
híbridas com programação linear inteira para otimização global, a extensão do
uso para ponto-multiponto e a inclusão de nós de conversão de onda, e também a
consideração de rotas “ocultas”, como por exemplo em \textit{swaps} feitos com
outras operadoras.

Assim, o artigo contribui para o entendimento
conceitual do problema de roteamento físico em redes ópticas e oferece um ponto
de partida para o desenvolvimento de soluções automatizadas mais complexas e
aplicáveis a cenários reais de backbone e acesso.


