\section{Fundamentação Teórica} \label{sec:fundaments}

Nesta seção discutem-se os principais conceitos relacionados às redes ópticas e aos algoritmos de roteamento.

\subsection{Elementos de uma Rede Óptica}
Uma rede óptica é composta por diferentes elementos físicos:
\begin{itemize}
    \item \textbf{Cabos ópticos:} compostos por múltiplos tubos contendo fibras.
    \item \textbf{Fibras ópticas:} meio físico de transmissão de sinais luminosos.
    \item \textbf{Caixa de emenda:} ponto de junção e proteção de fibras.
    \item \textbf{Caixa de atendimento:} permite a derivação de fibras para usuários finais.
\end{itemize}
Cada elemento possui custos associados (de instalação, operação e manutenção) que influenciam no peso das arestas do grafo.

\subsection{Algoritmos de Roteamento}
O problema de roteamento em redes pode ser resolvido por diferentes algoritmos:
\begin{itemize}
    \item \textbf{Dijkstra:} encontra o caminho de menor custo em grafos com pesos não negativos.
    \item \textbf{Bellman-Ford:} lida com arestas de peso negativo, mas com maior complexidade temporal.
\end{itemize}
Para redes ópticas, em que os custos são sempre não negativos (distância, atenuação, ou custo de instalação), o algoritmo de Dijkstra é mais adequado.

\subsection{Programação Linear e Complexidade}
O roteamento de fibras pode ser formulado como um problema de \textbf{programação linear inteira},
onde variáveis binárias indicam se uma fibra ou enlace é utilizado.
Entretanto, tais problemas são NP-difíceis, devido à combinação exponencial de caminhos possíveis.
Por isso, heurísticas como Dijkstra são práticas e eficientes em instâncias reais.

\subsection{Roteamento e Alocação de Banda}
Além da escolha do caminho, a rede precisa alocar espectro e largura de banda.
O problema de \textit{Routing and Wavelength Assignment} (RWA) é conhecido por sua complexidade computacional,
reforçando a necessidade de algoritmos heurísticos e aproximativos.
