\section{Introdução}

As redes ópticas representam a espinha dorsal da conectividade global,
responsáveis pela transmissão de grandes volumes de dados com alta
confiabilidade. Um dos desafios centrais no planejamento e operação dessas
redes é o roteamento de caminhos físicos, isto é, a definição de quais fibras
ópticas serão utilizadas para conectar dois pontos geográficos distintos. Este
artigo propõe o uso do algoritmo de Dijkstra para realizar o roteamento ótimo
de caminhos físicos em uma topologia representada por grafos ponderados.
