\section{Introdução}

Atualmente, as redes ópticas representam a espinha dorsal da conectividade
global, responsáveis pela transmissão de grandes volumes de dados em torno do
globo terrestre via cabos aéreos e submarinos. Embora as normas e os desafios
iniciais das implementações de redes ópticas pioneiras tenham atigindo um platô
de estabilidade na última década, devido a expansão desse tipo de rede dos
backbones para as redes de acesso, ainda há uma série de desafios relacionados
a operação e a implementação desse tipo de rede.

Um dos desafios centrais no planejamento e operação dessas redes é o roteamento
de caminhos físicos, isto é, a definição de quais fibras ópticas serão
utilizadas para conectar dois pontos geográficos distintos. Este artigo propõe
desenvolver um algoritmo heurístico para o roteamento de uma fibra dado uma
rede em funcionamento. Esse artigo é organizado no seguinte formato: na Seção
\ref{sec:fundaments} aborda-se os elementos de rede relevantes para o estudo
proposto, os algoritmos de base e o que já existe de estudos acadêmicos
relevantes próximo do presente artigo na Seção \ref{sec:problem} define-se e
delimita-se o problema de roteamento, na Seção \ref{sec:algorithm}
desenvolve-se o algoritmo e, por fim, as considerações finais são dadas em
\ref{sec:conclusion}.

