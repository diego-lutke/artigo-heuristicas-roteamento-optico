\section{Contexto e Diagnóstico} \label{sec:context}

Nas operadoras de telecomunicações, a operação de redes ópticas envolve a
manutenção de inventários físicos detalhados, com informações sobre cabos,
fibras e equipamentos. Tradicionalmente, o roteamento e a alocação de fibras
são realizados por sistemas comerciais de gerenciamento de rede (NMS —
\textit{Network Management Systems}) que utilizam dados georreferenciados e
inventários para automatizar o processo \cite{livroprofessor}.

Entretanto, no contexto acadêmico, é possível modelar o problema por meio de
representações em grafos, onde vértices representam nós de rede e arestas
representam enlaces ópticos. Cada enlace possui atributos, como comprimento,
número de fibras e estado de ocupação. Essa abstração permite aplicar
algoritmos clássicos de otimização, como Dijkstra e Bellman-Ford, adaptados à
realidade das redes ópticas \cite{bazaraa2011linear}.

O desafio, portanto, consiste em definir uma heurística que associe o menor
caminho físico e a disponibilidade de fibras livres, evitando bloqueios e
sobrecargas, e permitindo futura expansão para o problema completo de
\textit{Routing and Wavelength Assignment} (RWA) \cite{artigorwa}.
