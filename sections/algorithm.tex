\section{Estudo de Caso} \label{sec:algorithm}

% Considere uma rede óptica modelada como um grafo $G=(V,E)$, onde $V$ representa
% os nós (caixas de emenda ou atendimento) e $E$ representa enlaces
% (cabos/fibras). Cada aresta $e \in E$ possui um peso $w(e)$ correspondente ao
% custo físico (distância, atenuação ou custo de implantação). O objetivo é
% determinar o caminho de menor custo entre um nó origem $s$ e um nó destino $t$.
%
\subsection{Estrutura da Heurística Proposta}

O algoritmo proposto é dividido em duas camadas: (1) cálculo do caminho mais
curto, implementado a partir do algoritmo de Dijkstra e responsável por
identificar o percurso de menor custo entre origem e destino; (2) escolha e
alocação das fibras, após definido o caminho, onde o algoritmo verifica a
disponibilidade de fibras livres em cada enlace e seleciona a melhor fibra.

A melhor fibra é escolhida baseada em três modos (1) coletando fibra em tubo
conectado, (2) coletando uma fibra equivalente $f_e = (F_{>} / {F_{<}}) \bmod
  f_{a}$, onde $f_{e} \leq F_{dest}$ e (3) coletando a primeira
  fibra vaga de forma sequencial, isto é, desconsiderando tubos.

A sequência de alocação das fibras pode ser definida antes da execução do
algoritmo a depender do tipo de sinal trafegado e dos padrões definidos pelo
projetista, por meio de uma chave lógica. Um exemplo de uso seria utilizar o
algoritmo para alocar as fibras de menor índice para backbones e as fibras de
maior índice para redes PON.

O uso de memória deve ser considerado também para redes de grande porte.
Supondo que seja selecionado um número elevado de melhores rotas, a
complexidade de espaço será igual a $O(n)$ do somatório da rota com a maior
quantidade de conexões de tubo, visto que a pilha das conexões deve ser
descartadas após a análise de cada rota (linha 7 do Algoritmo
\ref{alg:the_alg}).

Interessante também considerar o armazenamento da última fibra usada nos modos
1 ou 2. Caso a fibra atual não seja uma fibra similar a armazenada, pode-se
continuar com a fibra armazenada. Esse tipo de situação leva ao exemplo
mostrado na Figura \ref{fig:fluxo_reduzido}.

\subsection{Pseudocódigo Simplificado}

\begin{algorithm}[H]
\floatname{algorithm}{Algoritmo}
\caption{Seleção das melhores fibras}
\label{alg:the_alg}
\begin{algorithmic}[1]
\State $\mathcal{R} \gets$ selecionar $n$ melhores rotas
\ForAll{$r \in \mathcal{R}$}
    \State $\mathcal{T}_r \gets$ coletar todos os tubos dos cabos da rota $r$
    \State $\mathcal{N}_r \gets$ encontrar todas as caixas (nós) entre os tubos de $\mathcal{T}_r$
    \ForAll{$n \in \mathcal{N}_r$}
        \If{existem tubos encaixados em $n$ com base nas fusões existentes}
            \State adicionar $n$ à pilha $\mathcal{T}_{c}$
        \EndIf
    \EndFor

    \LineComment{Iteração pela rota com cabos e nós em sequência}
    \ForAll{($c, n$) em sequência na rota $r$}
        \If{$c$ é o cabo de origem}
            \State $f \gets$ primeira fibra vaga em $c$
            \IfNot{$f$}
                \State cancelar iteração e tentar próxima rota
                \State \textbf{continue}
            \EndIf
        \Else
            \State $c_{a} \gets$ cabo anterior
            \State $f_{a} \gets$ fibra utilizada em $c_{a}$

            \LineComment{Três modos para encontrar a fibra vaga do cabo seguinte}
            \If{fibra $f_{a}$ pertence a um tubo encaixado}
                \State seguir na mesma fibra $f_{a}$ se estiver vaga
            \ElsIf{fibra equivalente $f_{e}$ estiver vaga}
                \State seguir com $f_{e}$
            \Else
                \State escolher a primeira fibra vaga, da mesma forma como o primeiro cabo
            \EndIf
        \EndIf
    \EndFor

    \State retornar cabos e fibras utilizados
\EndFor
\State \textbf{fim do algoritmo}
\end{algorithmic}
\end{algorithm}

\subsection{Resultados Esperados}

A heurística permite identificar rotas viáveis com uso eficiente das fibras,
reduzindo bloqueios e maximizando a utilização de recursos ópticos. Em cenários
com múltiplos caminhos equivalentes, o algoritmo pode ser estendido para
priorizar critérios adicionais, como distância física, tempo de
restabelecimento ou confiabilidade. Embora ainda não tenha sido implementada em
um ambiente de simulação, a estrutura proposta fornece base sólida para
experimentos futuros, seja em ferramentas de modelagem (como o Cisco Modeling
Labs) ou em linguagens de programação voltadas a grafos (como Python/NetworkX) \cite{alwayn2004optical}.

A alocação das fibras nas bandejas das caixas é feita de forma sequencial 
dando prioridade a tubos já acomodados. Isso traz eficiência nas operações de
rede e rapidez no diagnóstico de manutenções, visto que há um aproveitamento e
otimização dos recursos utilizados.

