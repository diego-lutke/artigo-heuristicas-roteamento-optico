\section{Conclusão}

Este artigo apresentou uma abordagem para o roteamento físico em redes ópticas
utilizando o algoritmo de Dijkstra. A fundamentação teórica mostrou a
relevância de representar redes como grafos ponderados e a comparação com
outros algoritmos. Ainda que problemas de roteamento óptico pertençam à classe
NP-difícil em sua formulação completa (com restrições de alocação de banda e
espectro), heurísticas como Dijkstra permitem soluções práticas e eficientes.
Como trabalhos futuros, recomenda-se a incorporação de restrições adicionais
(capacidade, espectro, múltiplos caminhos redundantes) e a aplicação de
técnicas híbridas com programação linear inteira para otimização global.
