\begin{resumo} 
  Este artigo propõe o desenvolvimento de um algoritmo heurístico para roteamento
  de fibras em redes ópticas, com o objetivo de otimizar a escolha de rotas e
  alocação de fibras disponíveis. A proposta parte de uma modelagem prática
  considerando a posição física dos tubos e fibras nos nós, sugerindo o algoritmo
  de Dijkstra para determinação das rotas e a proposição de um algoritmo
  adicional para a escolha das fibras. O estudo discute os princípios teóricos
  envolvidos, a estrutura do algoritmo proposto e as perspectivas de aplicação em
  redes ópticas de grande porte. Como resultado esperado, busca-se demonstrar a
  eficiência da abordagem em termos de simplicidade, aplicabilidade e potencial
  de integração com sistemas de gestão de rede.
\end{resumo}

\begin{abstract}
  This article proposes the development of a heuristic algorithm for fiber routing
  in optical networks, with the aim of optimizing route selection and
  allocation of available fibers. The proposal is based on practical modeling
  considering the physical position of the tubes and fibers in the nodes, suggesting the
  Dijkstra algorithm for determining routes and proposing an additional algorithm
  for fiber selection. The study discusses the theoretical principles
  involved, the structure of the proposed algorithm, and the prospects for application in
  large optical networks. As an expected result, it seeks to demonstrate the
  efficiency of the approach in terms of simplicity, applicability, and potential
  for integration with network management systems.
\end{abstract}
