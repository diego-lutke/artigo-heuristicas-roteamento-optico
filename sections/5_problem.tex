\section{Definição do Problema} \label{sec:problem}

Para delimitar o escopo do problema, primeiro: elementos ativos não serão
considerados (ex.: terminal de linha óptico); segundo: uma fibra será
considerada ocupada (isto é, alocada) caso haja um comprimento de onda
associado a ela, ao longo desse artigo, vamos considerar apenas um comprimento
de onda por alocação (multiplexaćão por divisão de onda não será considerado);
terceiro: desconsidera-se atenuação; quarto: desconsidera-se o uso de
splitters, onde um mesmo comprimento de onda em uma fibra é "espalhado" para
$n$ fibras em um nó.

Fazendo um palalelo com \cite{artigorwa} e \cite{zang2000review}, podemos notar
a menção de "conversores ópticos", em uma aplicaćão mais genérica. O presente
artigo é um "subconjunto" do tema de alocação e roteamento de onda tratado nas
referências citadas.

- como uma rede nova é ocupada? De cima para baixo, ou de baixo para cima?
- qual seria a solução ótima?
- cabo de x para n para x fibras (fazer uma figura?);
- fatorial das combinações;
- tubos abertos vs. sangria;
- quando fazer manobras;
- fazer comparativo com o problema de otimização de fluxo máximo;

Considere uma rede óptica modelada como um grafo $G=(V,E)$,
onde $V$ representa os nós (caixas de emenda ou atendimento) e $E$ representa enlaces (cabos/fibras).
Cada aresta $e \in E$ possui um peso $w(e)$ correspondente ao custo físico (distância, atenuação ou custo de implantação).
O objetivo é determinar o caminho de menor custo entre um nó origem $s$ e um nó destino $t$.
