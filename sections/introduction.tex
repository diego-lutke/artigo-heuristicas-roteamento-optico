\section{Introdução}

As redes ópticas representam a espinha dorsal das infraestruturas de
telecomunicações modernas, oferecendo alta capacidade de transmissão e
confiabilidade. Na medida em que a demanda por largura de banda cresce,
torna-se essencial otimizar o uso dos recursos físicos dessas redes, incluindo
o roteamento e a alocação de fibras ópticas disponíveis.

Um desses desafios é o roteamento de caminhos físicos, isto é, a definição de
quais fibras ópticas devem ser utilizadas para interligar dois pontos de
acesso. Este artigo propõe um algoritmo heurístico para o roteamento de uma
fibra dado uma rede em funcionamento.

Para isso, o artigo é estruturado da seguinte forma: na Seção \ref{sec:context}
é abordado como o roteamento em redes ópticas pode ser modelado academicamente
como um problema de grafos, aplicando algoritmos de otimização para encontrar
caminhos físicos com fibras disponíveis e evitar bloqueios na rede, na Seção
\ref{sec:fundaments} aborda-se os elementos de rede relevantes para o estudo
proposto, os algoritmos de base e o que já existe de estudos acadêmicos
relevantes próximo do presente artigo na Seção \ref{sec:problem} define-se e
delimita-se o problema de roteamento, na Seção \ref{sec:algorithm} propoe-se o
algoritmo para escolha das fibras e, por fim, as considerações finais são dadas
na Seção \ref{sec:conclusion}.

