\documentclass[aspectratio=169,xcolor=dvipsnames]{beamer}
\usetheme{SimplePlus}

\usepackage{hyperref}
\usepackage{graphicx} % Allows including images
\usepackage{booktabs} % Allows the use of \toprule, \midrule and \bottomrule in tables

\title{SimplePlus Beamer Theme}
\subtitle{Subtitle}
\author{Diego M. A. Lütke}
\institute
{
    Department of Computer Science and Information Engineering \\
    National Taiwan University
}
\date{\today} % Date, can be changed to a custom date

\begin{document}

\begin{frame}
    % Print the title page as the first slide
    \titlepage
\end{frame}

\begin{frame}{Overview}
    % Throughout your presentation, if you choose to use \section{} and
    % \subsection{} commands, these will automatically be printed on this slide
    % as an overview of your presentation
    \tableofcontents
\end{frame}

\section{First Section}

\section{Motivação e Objetivo}
\begin{frame}{Motivação}
  \begin{itemize}
    \item Redes ópticas: alta capacidade e papel crítico em infraestruturas de telecom.
    \item Problema prático: otimizar roteamento físico e alocação de fibras para evitar bloqueios.
    \item Restrições operacionais: inventário físico, bandejas e fusões em nós.
  \end{itemize}
  \note{(50s) Explicar por que alocação física de fibras é importante: manutenção, restauração, organização das bandejas e impacto operacional.}
\end{frame}

\begin{frame}{Objetivo do Artigo}
  \begin{itemize}
    \item Propor uma heurística prática para roteamento físico — caminho + escolha de fibra.
    \item Base: cálculo do menor caminho (Dijkstra) + algoritmo de seleção de fibras com três modos.
    \item Foco em simplicidade, aplicabilidade e integração com NMS.
  \end{itemize}
  \note{(40s) Reforçar que o objetivo não é resolver RWA completo, mas um subconjunto prático do problema.}
\end{frame}

\section{Contexto e Modelagem}
\begin{frame}{Modelagem como Grafo}
  \begin{itemize}
    \item Nós: Caixas de Emenda (CE) e Caixas de Atendimento (CA).
    \item Arestas: cabos ópticos (com atributos: comprimento, número de fibras, ocupação).
    \item Abstração permite uso de algoritmos de caminho mínimo (Dijkstra).
  \end{itemize}
  \note{(40s) Mostrar o paralelismo entre grafo e rede física. Explicar que cada enlace tem um vetor de fibras e estado binário (ocupada/vaga).}
\end{frame}

\begin{frame}{Estado da Arte}
  \begin{itemize}
    \item Problemas exatos (programação inteira) são NP-difíceis para instâncias reais.
    \item Heurísticas (Dijkstra, variantes, alg. aproximativos) são práticas para operações em redes reais.
    \item RWA (Routing \& Wavelength Assignment) é o problema mais geral; este trabalho reduz escopo para 1 $\lambda$ por alocação.
  \end{itemize}
  \note{(45s) Comentar sobre trade-off: precisão ótima vs. praticidade/tempo de execução.}
\end{frame}

\section{Escopo e Simplificações}
\begin{frame}{Delimitação do Problema}
  \begin{itemize}
    \item Elementos ativos desconsiderados (sem TLA/OLTs/etc.).
    \item Uma alocação = 1 comprimento de onda por fibra (sem splitters ou multiplexação multi-$\lambda$).
    \item Atenuação, splitters e conversores ópticos não modelados.
    \item Considera-se ocupada a fibra que já tem um $\lambda$ alocado.
  \end{itemize}
  \note{(40s) Justificar simplificações: facilitar heurística, foco em operações comuns de inventário físico.}
\end{frame}

%
% \begin{frame}[fragile]{Pseudocódigo (resumo)}
% \begin{algorithm}[H]
% \caption*{Seleção das melhores fibras — versão condensada}
% \begin{algorithmic}[1]
% \State $\mathcal{R} \gets$ selecionar $n$ melhores rotas (Dijkstra)
% \ForAll{$r \in \mathcal{R}$}
%   \State coletar tubos e nós da rota $r$
%   \ForAll{enlace $(c,n)$ ao longo de $r$}
%     \If{origem}
%       \State $f \gets$ primeira fibra vaga em $c$
%       \IfNot{$f$} \State \textbf{continue} (próxima rota)
%       \EndIf
%     \Else
%       \State $f_a \gets$ fibra anterior
%       \If{mesmo tubo e $f_a$ vaga} \State $f \gets f_a$
%       \ElsIf{fibra equivalente $f_e$ vaga} \State $f \gets f_e$
%       \Else \State $f \gets$ primeira vaga
%       \EndIf
%     \EndIf
%   \EndFor
%   \State retornar rota e fibras se sucesso
% \EndFor
% \end{algorithmic}
% \end{algorithm}
%   \note{(80s) Ler rapidamente o pseudocódigo, destacando pontos de falha (quando cancelar rota) e a pilha usada para tubos encaixados.}
% \end{frame}
%
\begin{frame}{Complexidade e Uso de Memória}
  \begin{itemize}
    \item Cálculo das $n$ melhores rotas: custo depende da implementação de Dijkstra / K-shortest.
    \item Análise de cada rota: espaço proporcional ao comprimento da maior rota (pilha de conexões) $\Rightarrow O(n)$ em termos práticos.
    \item A escolha de fibra é linear na quantidade de enlaces da rota (verificação por enlace).
  \end{itemize}
  \note{(40s) Comentar que heurística evita combinatorial blow-up da formulação exata, por isso é prática para redes de grande porte.}
\end{frame}

\section{Resultados Esperados e Aplicabilidade}
\begin{frame}{Resultados Esperados}
  \begin{itemize}
    \item Identificar rotas viáveis com melhor aproveitamento de fibras.
    \item Redução de bloqueios e manutenção do espelhamento físico nas bandejas.
    \item Simplicidade de integração com NMS e implementação em ferramentas como NetworkX/Cisco CML.
  \end{itemize}
  \note{(40s) Explicar cenários de uso: provisionamento automatizado, planejamento de rota em restauração, testes em ambiente simulado.}
\end{frame}

\begin{frame}{Limitações}
  \begin{itemize}
    \item Sem testes empíricos no artigo (falta validação em simulação/produção).
    \item Não trata múltiplos comprimentos de onda por fibra nem atenuação.
    \item Estratégia heurística pode não ser ótima em cenários altamente carregados.
  \end{itemize}
  \note{(30s) Ser crítico: indicar quando a heurística pode falhar e necessidade de comparação com outras abordagens.}
\end{frame}

\section{Conclusão e Trabalhos Futuros}
\begin{frame}{Conclusão}
  \begin{itemize}
    \item Heurística integra cálculo de menor caminho com regras práticas de alocação de fibra.
    \item Boa relação entre simplicidade operacional e aplicabilidade prática.
    \item Serve como base para implementações e experimentos futuros.
  \end{itemize}
  \note{(30s) Reforçar contribuição conceitual e utilidade prática como ponto de partida.}
\end{frame}

\begin{frame}{Trabalhos Futuros}
  \begin{itemize}
    \item Implementação e avaliação em simulador (CML, NetworkX, OMNeT++).
    \item Extensão para múltiplos $\lambda$ por fibra (RWA completo) e inclusão de atenuação/distância.
    \item Comparação com heurísticas/algoritmos meta-heurísticos (GA, GRASP, etc.).
    \item Integração com dados reais de inventário e políticas de operação.
  \end{itemize}
  \note{(40s) Sugerir métricas para avaliação: taxa de bloqueio, utilização de fibra, tempo de alocação, custo de manutenção.}
\end{frame}

\begin{frame}{Blocks of Highlighted Text}
    In this slide, some important text will be \alert{highlighted} because it's important. Please, don't abuse it.

    \begin{block}{Block}
        Sample text
    \end{block}

    \begin{alertblock}{Alertblock}
        Sample text in red box
    \end{alertblock}

    \begin{examples}
        Sample text in green box. The title of the block is ``Examples".
    \end{examples}
\end{frame}

\begin{frame}{Multiple Columns}
    \begin{columns}[c] % The "c" option specifies centered vertical alignment while the "t" option is used for top vertical alignment

        \column{.45\textwidth} % Left column and width
        \textbf{Heading}
        \begin{enumerate}
            \item Statement
            \item Explanation
            \item Example
        \end{enumerate}

        \column{.45\textwidth} % Right column and width
        Lorem ipsum dolor sit amet, consectetur adipiscing elit. Integer lectus
        nisl, ultricies in feugiat rutrum, porttitor sit amet augue. Aliquam ut
        tortor mauris. Sed volutpat ante purus, quis accumsan dolor.

    \end{columns}
\end{frame}

\section{Second Section}

\begin{frame}{Table}
    \begin{table}
        \begin{tabular}{l l l}
            \toprule
            \textbf{Treatments} & \textbf{Response 1} & \textbf{Response 2} \\
            \midrule
            Treatment 1         & 0.0003262           & 0.562               \\
            Treatment 2         & 0.0015681           & 0.910               \\
            Treatment 3         & 0.0009271           & 0.296               \\
            \bottomrule
        \end{tabular}
        \caption{Table caption}
    \end{table}
\end{frame}

\begin{frame}{Theorem}
    \begin{theorem}[Mass--energy equivalence]
        $E = mc^2$
    \end{theorem}
\end{frame}

\begin{frame}{Figure}
    Uncomment the code on this slide to include your own image from the same directory as the template .TeX file.
    %\begin{figure}
    %\includegraphics[width=0.8\linewidth]{test}
    %\end{figure}
\end{frame}

\begin{frame}[fragile] % Need to use the fragile option when verbatim is used in the slide
    \frametitle{Citation}
    An example of the \verb|\cite| command to cite within the presentation:\\~
\end{frame}

\nocite{*}

\setbeamertemplate{bibliography item}[triangle]
\begin{frame}{References}
    \footnotesize
    \bibliographystyle{ieeetr}
    \bibliography{../util/references.bib}
\end{frame}

\end{document}
